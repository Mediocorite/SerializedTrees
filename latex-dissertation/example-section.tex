\section{Example section}
Some interesting ideas are given in Ref.~\cite{turing1950computing}.  A list of comparison operators are given in Table~\ref{table:comparison-operators}.  The colour crest is given in Figure~\ref{figure:colour-crest}.  A source code listing is given in Listing~\ref{listing:program}.

% An example table.
\begin{table}[h!!]
  \begin{center}
    \caption{Python comparison operators.}
    \label{table:comparison-operators}
    \begin{tabular}{l|l} \hline
      \textbf{Operator} & \textbf{Name}\\
      \hline
      \texttt{==} & Equal to\\
      \texttt{!=} & Not equal to\\
      \texttt{>} & Greater than\\
      \texttt{<} & Less than\\
      \texttt{>=} & Greater than or equal to\\
      \texttt{<=} & Less than or equal to\\ \hline
    \end{tabular}
  \end{center}
\end{table}

% An example figure.
\begin{figure}[h!!]
  \begin{center}
    \includegraphics[scale=0.2]{strath_fullcolour.png}
    \caption{The University of Strathclyde colour crest.}
    \label{figure:colour-crest}
  \end{center}
\end{figure}

% An example source code listing.
\begin{singlespacing}
\begin{lstlisting}[language=python,caption={Demonstrating source code listing.},label=listing:program]
def fun():
    """
    A function to print a text string.
    """
    print("This is a test")


fun()
\end{lstlisting}
\end{singlespacing}

\lipsum[1-2]
\subsection{A subsection}
\lipsum[1-2]
\subsubsection{A subsubsection}
\lipsum[1-2]